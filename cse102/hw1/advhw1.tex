\documentclass[12pt]{article}

% This first part of the file is called the PREAMBLE. It includes
% customizations and command definitions. The preamble is everything
% between \documentclass and \begin{document}.

\usepackage[margin=1in]{geometry}  % set the margins to 1in on all sides
\usepackage{graphicx}              % to include figures
\usepackage{amsmath}               % great math stuff
\usepackage{amsfonts}              % for blackboard bold, etc
\usepackage{amsthm}                % better theorem environments
\usepackage{amssymb} 
\usepackage{mathptmx}


% various theorems, numbered by section

\newtheorem{thm}{Theorem}[section]
\newtheorem{lem}[thm]{Lemma}
\newtheorem{prop}[thm]{Proposition}
\newtheorem{cor}[thm]{Corollary}
\newtheorem{conj}[thm]{Conjecture}
\newtheorem{mydef}[thm]{Definition}

\begin{document}


\title{ CSE 102 Spring 2021\\
	Advanced Homework Assignment 1}

\author{Jaden Liu \\ 
University of California at Santa Cruz\\
Santa Cruz, CA 95064 USA }

\maketitle


\section{AdvHW1} 

\textbf{1. Prove that $\binom{2n}{n}=\Theta(\frac{4^n}{\sqrt{n}})$, where $\binom{m}{k}$ denotes the binomial coefficient $\binom{m}{k}=\frac{m!}{k!(m-k)!}$, for $(0\le k\le m)$.}\\
\begin{proof}
	By using $Stirling's\ Formula$, we can get:\cite{wiki}
	\begin{align*}
		\binom{2n}{n}&=\frac{(2n)!}{n!\cdot n!}\\
		&=\frac{\sqrt{2\pi 2n}\cdot(\frac{2n}{e})^{2n}}{(\sqrt{2\pi n}\cdot(\frac{n}{e})^n)^2}\\
		&=\frac{\sqrt{4\pi n}\cdot\frac{4^n\cdot n^{2n}}{e^{2n}}}{(\sqrt{2\pi n}\cdot\frac{n^n}{e^n})^2}\\
		&=\frac{4^n}{\sqrt{\pi n}}\\
	\end{align*}
Therefore, we can observe that:
\begin{align*}
	\frac{\sqrt{\pi}}{100}\cdot\frac{4^n}{\sqrt{\pi n}}\le&\frac{4^n}{\sqrt{\pi n}}\le\sqrt{\pi}\cdot\frac{4^n}{\sqrt{\pi n}}\\
	\frac{1}{100}\cdot\frac{4^n}{\sqrt{n}}&\le\frac{4^n}{\sqrt{\pi n}}\le\frac{4^n}{\sqrt{n}}\\
	\frac{1}{100}\cdot\frac{4^n}{\sqrt{n}}&\le\binom{2n}{n}\le\frac{4^n}{\sqrt{n}}
\end{align*}
Hence, we have $\binom{2n}{n}=\Theta(\frac{4^n}{\sqrt{n}})$.
\end{proof}



\bigskip


\begin{thebibliography}{wiki}
	\bibitem{wiki} Wikipidea, https://en.wikipedia.org/wiki/Stirling%27s_approximation
\end{thebibliography}
\end{document}
