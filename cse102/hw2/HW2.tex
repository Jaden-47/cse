\documentclass[12pt]{article}

% This first part of the file is called the PREAMBLE. It includes
% customizations and command definitions. The preamble is everything
% between \documentclass and \begin{document}.

\usepackage[margin=1in]{geometry}  % set the margins to 1in on all sides
\usepackage{graphicx}              % to include figures
\usepackage{amsmath}               % great math stuff
\usepackage{amsfonts}              % for blackboard bold, etc
\usepackage{amsthm}                % better theorem environments
\usepackage{amssymb} 
\usepackage{mathptmx}


% various theorems, numbered by section

\newtheorem{thm}{Theorem}[section]
\newtheorem{lem}[thm]{Lemma}
\newtheorem{prop}[thm]{Proposition}
\newtheorem{cor}[thm]{Corollary}
\newtheorem{conj}[thm]{Conjecture}
\newtheorem{mydef}[thm]{Definition}

\begin{document}


\title{ CSE 102 Spring 2021\\
	Homework Assignment 2}

\author{Jaden Liu \\ 
University of California at Santa Cruz\\
Santa Cruz, CA 95064 USA }

\maketitle


\section{HW2} 

\textbf{1. Use induction to prove that $\sum_{k=1}^{n}k^4=\frac{n(n+1)(6n^3+9n^2+n-1)}{30}$. }\\

\begin{proof}
	We use induction to prove this, let $P(n)$ be the equation above.
\begin{enumerate}
	\item \textbf{Base step:}\\
	$\sum_{k=1}^{1}k^4=\frac{1(1+1)(6+9+1-1)}{30}=1^4$, show that $P(1)$ is true.
	\item \textbf{Induction step:}\\
	Let $n\ge1$ and assume $P(n)$is true. That is, for this particular value of $n$, the
	equation holds. Then
	\begin{align*}
		\sum_{k=1}^{n+1}k^4&=\sum_{k=1}^{n}k^4+(n+1)^4\\
		&=\frac{n(n+1)(6n^3+9n^2+n-1)}{30}+(n+1)^4\\	
		&=\frac{(n+1)(6n^4+9n^3+n^2-n+30(n+1)^3)}{30}\\
		&=\frac{(n+1)(6n^4+9n^3+n^2-n+30n^3+90n^2+90n+30)}{30}\\
		&=\frac{(n+1)(6n^4+39n^3+91n^2+89n+30)}{30}\\
		&=\frac{(n+1)(n+2)(6n^3+27n^2+37n+15)}{30}\\
		&=\frac{(n+1)((n+1)+1)(6(n+1)^3+9(n+1)^2+(n+1)-1)}{30}
	\end{align*}
\qquad showing that $P(n+1)$ is true.\\
We conclude that $P(n)$ is true for all $n\ge1$.
\end{enumerate}
\end{proof}





\noindent \textbf{2. Let $T(n)$ be defined by the recurrence:
	\begin{align*}
			T(n)=\begin{cases}
			1 &n=1\\
			T(\lfloor\frac{n}{2}\rfloor)+n^2 &n\ge2
			\end{cases}
	\end{align*}\\
Use substitution method to show that $\forall n\ge1$: $T(n)\le\frac{4}{3}n^2$, and hence $T(n)=O(n^2)$.
}
\begin{proof}
	We use induction to prove this, let $P(n)$ be the inequality above.
	\begin{enumerate}
		\item \textbf{Base step:}\\
		We can easily observe that $T(1)=1\le\frac{4}{3}1^2$, show that $P(1)$ is true.
		\item \textbf{Induction step:}\\
		Let $n>1$ and assume for all $k$ in the range $1\le k<n$ that $P(k)$ is true. In particular when $k=\lfloor \frac{n}{2}\rfloor$, we have $T(\lfloor \frac{n}{2}\rfloor)\le \frac{4}{3}(\lfloor \frac{n}{2}\rfloor)^2$. We must show $T(n)\le \frac{4}{3}n^2$ is true.
		\begin{align*}
			T(n)&=T(\lfloor\frac{n}{2}\rfloor)+n^2\\
			&\le\frac{4}{3}(\lfloor \frac{n}{2}\rfloor)^2+n^2\\
			&\le\frac{4}{3}(\frac{n}{2})^2+n^2\\
			&=\frac{4}{3}n^2
		\end{align*}
	\end{enumerate}
Therefore, we conclude that $P(n)$ is true and $T(n)\le\frac{4}{3}n^2$ for all $n\ge1$.
\end{proof}






\noindent \textbf{3.}
\begin{proof}
		We use induction to prove this, let $P(n)$ be the equation above.
	\begin{enumerate}
		\item \textbf{Base step:}\\
		We can easily observe that $\sum_{k=1}^{1}kH_k=\frac{1}{2}1(1+1)H_1-\frac{1}{4}1(1-1)=H_1$, show that $P(1)$ is true.
		\item \textbf{Induction step:}\\
		Let $n\ge1$ and assume $P(n)$is true. That is, for this particular value of $n$, the
		equation holds. Then
		\begin{align*}
			\sum_{k=1}^{n+1}kH_k&=\sum_{k=1}^{n}kH_k+(n+1)H_{n+1}\\
			&=\frac{1}{2}n(n+1)H_{n}-\frac{1}{4}n(n-1)+(n+1)(H_{n}+\frac{1}{n+1})\\
			&=(\frac{1}{2}n+1)(n+1)H_{n}-\frac{1}{4}n(n-1)+1\\
			&=\frac{1}{2}(n+2)(n+1)H_{n}-\frac{1}{4}n(n+1)+\frac{1}{2}(n-2)\\
			&=\frac{1}{2}(n+2)(n+1)(H_{n}+\frac{1}{n+1})-\frac{1}{4}n(n+1)\\
			&=\frac{1}{2}(n+1)(n+2)H_{n+1}-\frac{1}{4}(n+1)n
		\end{align*}	
		\qquad showing that $P(n+1)$ is true.\\
		We conclude that $P(n)$ is true for all $n\ge1$.
	\end{enumerate}
\end{proof}

\noindent \textbf{4.}
\begin{proof}
	\begin{align*}
		T(n)&=T(n-1)+n\\
		&=T(n-2)+n+n-1\\
		&=T(n-3)+n+n-1+n-2\\
		&\vdots\\
		&=T(n-k)+kn-(1+2+\cdots+n-1)\\
		&=T(n-k)+kn-\frac{n(n-1)}{2}\\
	\end{align*}
This process must terminate when the recursion depth k reaches its maximum, i.e. when $n-k=1$.
\begin{align*}
	n-k&=1\\
	k&=n-1
\end{align*}
Thus for the recursion depth $k=n-1$ we have $T(n-k)=T(1)=1$, and hence the solution to the above recurrence is $T(n)=1+\frac{(n-1)n}{2}$.\\
Therefore, $T(n)=\Theta(n^2)$ since $n$ did not increase as rapidly as $n^2$ does.
\end{proof}





\noindent \textbf{5.}
\begin{proof}
	\begin{align*}
		T(n)&=T(\lfloor n/2\rfloor)+6\\
		&=T((\lfloor n/2\rfloor)/2)+12\\
		&=T(\lfloor n/2^2\rfloor/2)+18\\
		&\vdots\\
		&=6k+T(\lfloor n/2^k\rfloor)
	\end{align*}
	We seek the first integer k such that $\lfloor n/2^k\rfloor<15$, which is equivalent to $n/2^k<15$ .
	\begin{align*}
		n/15&<2^k\\
		k-1&\le\log_2(n/15)<k\\
		k-1&=\lfloor \log_2(n/15)\rfloor\\
		k&=\lfloor \log_2(n/15)\rfloor+1
	\end{align*}
	Thus, we have $T(n)=6(\lfloor \log_2(n/15)\rfloor+1)+9$.\\
	Hence $T(n)=\Theta(\log(n))$ since $\log(n)$ increase faster then constant does.
\end{proof}


\noindent \textbf{6.}
\begin{proof}[Solution for a]
	Observe that $a=3$, $b=3/2$, and $\log_ba=2.70951\dots<3$. Upon setting $\epsilon=3-\log_{\frac{3}{2}}3$ we have $\epsilon>0$, and therefore $f(n)=n^3=\Omega(n^{\log_{\frac{3}{2}}3+\epsilon})$. We can find $3f(\frac{n}{\frac{3}{2}})\le cf(n)$, since $\frac{3\cdot8\cdot n^3}{27}\le cn^3$, which is true when $\frac{24}{27}\le c<1$. Therefore, $T(n)=\Theta(n^3)$.
\end{proof}
\ \\
\begin{proof}[Solution for b]
	Observe that $a=2$, $b=3$, and $\log_ba=0.63092\dots >\frac{1}{2}$. Upon setting $\epsilon=\log_{3}2-\frac{1}{2}$ we have $\epsilon>0$, and therefore $f(n)=n^{\frac{1}{2}}=O(n^{\log_{3}2-\epsilon})$. Thus $T(n)=\Theta(n^{\log_32})$.
\end{proof}
\ \\
\begin{proof}[Solution for c]
	Observe that $a=5$, $b=4$, and $\log_ba=1.16096\dots<\lg\sqrt{5}$. Upon setting $\epsilon=\lg\sqrt{5}-\log_{4}5$ we have $\epsilon>0$, and therefore $f(n)=n^{\lg\sqrt{5}}=\Omega(n^{\log_{4}5+\epsilon})$. We can find $5f(n/4)\le cf(n)$ such c that $\frac{1}{5}\le c <1$. Thus $T(n)=\Theta(n^{\lg\sqrt{5}})$.
\end{proof}
\ \\
\begin{proof}[Solution for d]
	Observe that $a=3$, $b=\frac{5}{2}$, and $\log_ba=1.19897\dots >1$. Upon setting $\epsilon=0.1$ we have $\epsilon>0$. Let $g(n)=n^{\log_{\frac{5}{2}}3-\epsilon}$, now we need to prove $f(n)=n\log n=O(g(n))=O(n^{\log_{\frac{5}{2}}3-0.1})$:
	\begin{align*}
		\lim\limits_{n\to\infty}\frac{f(n)}{g(n)}&=\frac{n\log n}{n^{\log_{\frac{5}{2}}3-0.1}}\\
		&=\frac{\log n}{(\log_{\frac{5}{2}}3-0.1)n^{\log_{\frac{5}{2}}3-0.1-1}}\\
		&=\frac{n^{-1}}{(\log_{\frac{5}{2}}3-0.1)\cdot(\log_{\frac{5}{2}}3-0.1-1)n^{\log_{\frac{5}{2}}3-0.1-2}}\\
		&=\frac{n^{-1-(\log_{\frac{5}{2}}3-2.1)}}{(\log_{\frac{5}{2}}3-0.1)\cdot(\log_{\frac{5}{2}}3-0.1-1)}\\
		&=\frac{n^{1.1-(\log_{\frac{5}{2}}3-2.1)}}{(\log_{\frac{5}{2}}3-0.1)\cdot(\log_{\frac{5}{2}}3-0.1-1)} \\
		&=0\quad\textbf{since $\log_{\frac{5}{2}}3\approx1.19$, then the numerator is zero}
	\end{align*}
	Therefore $f(n)=n\log n=o(n^{\log_{\frac{5}{2}}3-1})$. Since $o$ is a subset of $O$, then $f(n)=n\log n=O(n^{\log_{\frac{5}{2}}3-1})$. Thus $T(n)=\Theta(n^{\log_{\frac{5}{2}}3})$.
	
\end{proof}

\begin{proof}[Solution for e]
	\ \\
	\begin{itemize}
		\item Case 1: $1\le a<16$\\
		We can observe that $\log_4a<2$, then setting $\epsilon=2-\log_4a$, we have $\epsilon>0$.  Therefore $f(n)=n^2=\Omega(n^{log_4a+\epsilon})$. Next we need to check $af(n/4)\le cf(n)$ for some $0<c<1$ and all sufficiently large $n$. This inequality says $a(n/4)^2\le cn^2$, which is true as long as $\frac{a}{16}\le c<1$. Therefore, $T(n)=\Theta(n^2)$.
		\item Case 2: $a=16$\\
		We can observe that $\log_4{16}=2$, then $f(n)=n^2=\Theta(n^{\log_4{16}})=\Theta(n^2)$, then $T(n)=\Theta(n^2\cdot\log(n))$\\
		\item Case 3: $a>16$\\
		We can observe that $\log_4a>2$, then setting $\epsilon=\log_{4}a-2$, we have $\epsilon>0$. Therefore $f(n)=n^2=O(n^{\log_4a-\epsilon})$. Thus $T(n)=\Theta(n^{\log_4a})$.
	\end{itemize}
\end{proof}
\bigskip


\begin{thebibliography}{prof}
	\bibitem{prof} Suresh Lodha, MasterTheoremPracticeProblemsSKL.pdf
	\bibitem{csulb} https://web.csulb.edu/~tebert/teaching/lectures/528/recurrence/recurrence.pdf
	\bibitem{pomona} https://cs.pomona.edu/~dkauchak/classes/algorithms/handouts/hw2solution.pdf
\end{thebibliography}


\end{document}
