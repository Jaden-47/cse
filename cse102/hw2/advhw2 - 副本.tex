\documentclass[12pt]{article}

% This first part of the file is called the PREAMBLE. It includes
% customizations and command definitions. The preamble is everything
% between \documentclass and \begin{document}.

\usepackage[margin=1in]{geometry}  % set the margins to 1in on all sides
\usepackage{graphicx}              % to include figures
\usepackage{amsmath}               % great math stuff
\usepackage{amsfonts}              % for blackboard bold, etc
\usepackage{amsthm}                % better theorem environments
\usepackage{amssymb} 
\usepackage{mathptmx}


% various theorems, numbered by section

\newtheorem{thm}{Theorem}[section]
\newtheorem{lem}[thm]{Lemma}
\newtheorem{prop}[thm]{Proposition}
\newtheorem{cor}[thm]{Corollary}
\newtheorem{conj}[thm]{Conjecture}
\newtheorem{mydef}[thm]{Definition}

\begin{document}


\title{ CSE 102 Spring 2021\\
	Advanced Homework Assignment 2}

\author{Jaden Liu \\ 
University of California at Santa Cruz\\
Santa Cruz, CA 95064 USA }

\maketitle


\section{AdvHW2} 

\textbf{5. Consider the recurrence relation: $T(1)=a$ and $T(n)=nT(n-1)+bn$
	for $n\ge2$. Prove, by induction, that for sufficiently large integer n, there
	exists two positive real constants $P$ and $Q$ such that $Pn!\le T(n)\le Qn!$.}\\
\begin{proof}
	First we assume $a,b>0$. To prove $Pn!\le T(n)\le Qn!$, we can equaivalently prove $T(n)=\Theta(n!)$.\\
	\begin{enumerate}
		\item Base step: when $n=1$,\\
			$T(1)=a$. Thus, we let $P=\frac{a}{2}$, $Q=2a$, then we can easily observe that $P\cdot1!\le T(1)\le Q\cdot1!$.
		\item Induction step:\\
			\begin{align*}
				T(n)&=nT(n-1)+bn\\
				&=n\cdot((n-1)T(n-2)+b(n-1))+bn\\
				&=n\cdot((n-1)((n-2)T(n-3)+b(n-2))+b(n-1))+bn\\
				&\vdots\\
				&=n!T(1)+b(n+n(n-1)+n(n-1)(n-2)+\cdots+n!)
			\end{align*}
	\end{enumerate}
	Let $f(n)=n!$, we compute the limit $\lim\limits_{n\to\infty}\frac{T(n)}{f(n)}$:
	\begin{align*}
		\lim\limits_{n\to\infty}\frac{T(n)}{f(n)}=&\lim\limits_{n\to\infty}\frac{n!T(1)+b(n+n(n-1)+n(n-1)(n-2)+\cdots+n!)}{n!}\\
		&=T(1)+\lim\limits_{n\to\infty}b(\frac{1}{(n-1)!}+\frac{1}{(n-2)!}+\frac{1}{(n-3)!}+\cdot+1)\\
		&=T(1)+b\lim\limits_{n\to\infty}(\sum_{i=1}^{n}\frac{1}{i!})\\
		&=T(1)+b(e-1)\\
		&=a+b(e-1)
	\end{align*}
	Since $0<a+b(e-1)<\infty$, then $T(n)=\Theta(n!)$. Thus, there must exist $P$ and $Q$ such that $Pn!\le T(n)\le Qn!$.
\end{proof}



\bigskip


\begin{thebibliography}{}
\end{thebibliography}
\end{document}
