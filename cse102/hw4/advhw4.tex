\documentclass[12pt]{article}

% This first part of the file is called the PREAMBLE. It includes
% customizations and command definitions. The preamble is everything
% between \documentclass and \begin{document}.

\usepackage[margin=1in]{geometry}  % set the margins to 1in on all sides
\usepackage{graphicx}              % to include figures
\usepackage{amsmath}               % great math stuff
\usepackage{amsfonts}              % for blackboard bold, etc
\usepackage{amsthm}                % better theorem environments
\usepackage{amssymb} 
\usepackage{mathptmx}
\usepackage{graphicx}
\usepackage{enumerate}
\usepackage{listings}
\usepackage{xcolor}
\usepackage{array}

% various theorems, numbered by section

\newtheorem{thm}{Theorem}[section]
\newtheorem{lem}[thm]{Lemma}
\newtheorem{prop}[thm]{Proposition}
\newtheorem{cor}[thm]{Corollary}
\newtheorem{conj}[thm]{Conjecture}
\newtheorem{mydef}[thm]{Definition}


\lstset{
	basicstyle          =   \sffamily,          
	keywordstyle        =   \bfseries,          
	commentstyle        =   \rmfamily\itshape,  
	stringstyle         =   \ttfamily,  
	flexiblecolumns,                
	numbers             =   left,   
	showspaces          =   false,  
	numberstyle         =   \fontsize{5}{skip},    
	showstringspaces    =   false,
	captionpos          =   t,      
	frame               =   lrtb,   
}

\lstdefinestyle{cpp}{
	language        =   cpp, 
	basicstyle      =   \fontsize{5}{skip},
	numberstyle     =   \fontsize{5}{skip},
	keywordstyle    =   \color{blue},
	keywordstyle    =   [2] \color{teal},
	stringstyle     =   \color{magenta},
	commentstyle    =   \color{red}\ttfamily,
	breaklines      =   true,   
	columns         =   fixed,  
	basewidth       =   0.5em,
}
\begin{document}


\title{ CSE 102 Spring 2021\\
	Advanced Homework Assignment 4}

\author{Jaden Liu \\ 
University of California at Santa Cruz\\
Santa Cruz, CA 95064 USA }

\maketitle


\section{AdvHW4} 

\textbf{Greedy Approximation to 0-1 Knapsack Problem: (up to 10 points: simple proof without external
	resources) Consider the 0-1 knapsack problem: Given n items with weights $w_1$, ..., $w_n$ and value $v_1$, ...$v_n$ and a total weight of W, where each $w_i$, $v_i$ and $B$ are positive integers, find a subset S of items that a thief would like to steal so that the total weight is smaller than W and the total value is maximum.}\\
Greedy strategy:

	1. Calculate the value of each item's unit weight vi/wi
	
	2. Pack items with the highest unit weight into the backpack.
	
	3. If total weight of the items in the backpack does not exceed give weight maximum W, then select items with the second highest value per unit weight. 
	
	4. Loop process 2 and 3 to pack as many backpacks as possible until the backpack is full.
	
	5. After the looping, we need to compare $v_{k+1}$ and $v_1+v_2+v_3+\cdots+v_k$ unless we could get wrong other. Our final answer would be $\max (v_{k+1},v_1+v_2+v_3+\cdots+v_k)$ for $1\le k\le n$

\begin{proof}
	First the we sort the knapsack's items in decreasing order of the value density $v_i/w_i$, then item $n_1$ has biggest value density with value $v_1$ and $w_1$. Assume we have the optimal total value $v_{max}$, then obviously $v_1+v_2+v_3+\cdots+v_k\le v_{max}$. Also we can have $v_{max}\le v_1+v_2+v_3+\cdots+v_k+v_{k+1}$. This situation is like partial Knapsack Problem. When we add partial $v_{k+1}$ to fill the pack completely. The pack has already been at the maximum value density, hence adding partial $v_{k+1}$ is equal to $v_{max}$. Not to mention adding whole $v_{k+1}$ is larger or equal to $v_{max}$. Then we have to two inequalities:\\
	\begin{equation*}
		\begin{cases}
		v_{max}\le v_1+v_2+v_3+\cdots+v_k+v_{k+1} \\  v_1+v_2+v_3+\cdots+v_k+v_{k+1}\le2\max(v_{k+1},v_1+v_2+v_3+\cdots+v_k)
		\end{cases}
	\end{equation*}
	It's easy to prove the second inequality if we consider $v_{k+1}=a$, $v_1+v_2+v_3+\cdots+v_k=b$, then $\max(a,b)\ge a$ and $\max(a,b)\ge b$. Thus $2\max(a,b)\ge a+b$.\\
	Combine these two inequalities, we can get $v_{max}\le2\max(v_{k+1},v_1+v_2+v_3+\cdots+v_k)$. Thus the solution of our greedy algorithm is at least $\frac{V_{max}}{2}$.\\
	
	For further convenience, I want to change the name "$v_{max}$" to "$v_{optimal}$" or "$v_{opt}$", and name the solution of our greedy algorithm "max($v_k+1,v_1+v_2+..v_k$)" as "$v_{max}$".
	
	1. It's hard to compare $v_{max}$ and $v_{opt}$. Therefore, we need to come up with a "bridge" to connect these two variables. It's natural to think of $\max(a,b) \ge a$ or $b$. In order to get the coefficient "2" in the final result, we can easily come up with inequality that $2*max(a,b) \ge a$+$b$. Then we find a+b, which is "$v_1+v_2+..+v_{k+1}$" is always larger than $v_{opt}$.
	
	2. 0-1 Knapsack problem is different from general Knapsack problem mainly because of its element divisibility. In general Knapsack problem, for example, we have fulfilled the pack with $v_1$ to $v_k$ with only $w_{rem}$. However, we have $w_{k+1}$ for the last element which cannot be packed in the bag.  Thus, we choose $w_{rem}/ w_{k+1}$ of the last element,   so the total value is $v_1+...+v_{k+1}*\frac{w_{rem}}{w_{k+1}}$. It's easy to prove $v_{max} = v_{opt}$ in this situation since we have chosen from most valuable to least valuable to let the final pack density reaching maximum. However, in the 0-1 Knapsack problem, the elements are not divisible. Hence, $v_1+..+v_{k+1}$ is always larger to $v_{opt}$.
\end{proof}



\end{document}
