\documentclass[12pt]{article}

% This first part of the file is called the PREAMBLE. It includes
% customizations and command definitions. The preamble is everything
% between \documentclass and \begin{document}.

\usepackage[margin=1in]{geometry}  % set the margins to 1in on all sides
\usepackage{graphicx}              % to include figures
\usepackage{amsmath}               % great math stuff
\usepackage{amsfonts}              % for blackboard bold, etc
\usepackage{amsthm}                % better theorem environments
\usepackage{amssymb} 
\usepackage{mathptmx}
\usepackage{enumerate}
\usepackage{listings}
\usepackage{xcolor}
\usepackage{forest}
\usepackage{tabularx}  

% various theorems, numbered by section

\newtheorem{thm}{Theorem}[section]
\newtheorem{lem}[thm]{Lemma}
\newtheorem{prop}[thm]{Proposition}
\newtheorem{cor}[thm]{Corollary}
\newtheorem{conj}[thm]{Conjecture}
\newtheorem{mydef}[thm]{Definition}
\lstset{
	basicstyle          =   \sffamily,          
	keywordstyle        =   \bfseries,          
	commentstyle        =   \rmfamily\itshape,  
	stringstyle         =   \ttfamily,  
	flexiblecolumns,                
	numbers             =   left,   
	showspaces          =   false,  
	numberstyle         =   \fontsize{5}{skip},    
	showstringspaces    =   false,
	captionpos          =   t,      
	frame               =   lrtb,   
}

\lstdefinestyle{cpp}{
	language        =   cpp, 
	basicstyle      =   \fontsize{5}{skip},
	numberstyle     =   \fontsize{5}{skip},
	keywordstyle    =   \color{blue},
	keywordstyle    =   [2] \color{teal},
	stringstyle     =   \color{magenta},
	commentstyle    =   \color{red}\ttfamily,
	breaklines      =   true,   
	columns         =   fixed,  
	basewidth       =   0.5em,
}
\begin{document}


\title{ CSE 102 Spring 2021\\
	Quiz Reflection 5}

\author{Jaden Liu \\ 
University of California at Santa Cruz\\
Santa Cruz, CA 95064 USA }

\maketitle


\section{Quiz 5} 
\begin{proof}[Solution for q2]
	I read the problem wrong. What we need to compute is the order of the elemenet we get in the table instead of the real value in this table. Thus I compute the whole problem wrong.    \begin{tabular}{rrrr}
		0     & 1     & 2     & 3 \\
		1     & 3     & 2     & 1 \\
		2     & 6     & 5     & 4 \\
		3     & 9     & 8     & 7 \\
		4     & 12    & 11    & 10 \\
		5     & 15    & 14    & 13 \\
	\end{tabular}\\
	The value we needed is the right bottom cell 13.\\
	To get 10 we need the cell 7\&8. It's just top and top left cell in each iteration.
\end{proof}

\bigskip



\end{document}
